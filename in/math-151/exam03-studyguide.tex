::: Exam 3 Studyguide

% 1
Determine convergence or divergence, including absolute/conditional convergence
if applicable. State clearly what test is being used.
\[\sum_{n=1}^\infty \frac{n+1}{3n+1}\]
===
(Divergence test) \(\lim=\sfrac{1}{3}\) which \(\neq 0\) and therefore diverges.

---

% 2
Determine convergence or divergence, including absolute/conditional convergence
if applicable. State clearly what test is being used.
\[\sum_{n=1}^\infty 3{(\pi)}^{-n}\]
===
(Geometric series)
The series has the ratio \(r=\sfrac{1}{\pi}\), which because \(|r|<0\)
converges. The series converges to:
\begin{gather*}
a_1=\sfrac{3}{\pi};\ r=\sfrac{1}{\pi}\\
S_n=\frac{\sfrac{3}{\pi}}{1-\sfrac{1}{\pi}}\approx 1.4008
\end{gather*}

---

% 3
Determine convergence or divergence, including absolute/conditional convergence
if applicable. State clearly what test is being used.
\[\sum_{n=1}^\infty \frac{{(-1)}^n 3}{2n+1}\]
===
\begin{enumerate}
  \item (Check for absolute convergence using L.C.T.) Comparing against
    \(\sum_{n=1}^\infty \frac{1}{n}\) (divergent p-series):
    \[\lim_{n\to\infty}\left|\frac{3}{2n+1}\cdot\frac{n}{1}\right|=\frac{3}{2}\]
    Which is both finite and positive, and so diverges similarly. We need to
    check for conditional convergence.
  \item (A.S.T.)
    \begin{enumerate}
      \item \(\displaystyle \lim_{n\to\infty}\frac{{(-1)}^n
        3}{2n+1}=0\ \checkmark\)
      \item \(a_{n+1}<a_n\ \checkmark\)
    \end{enumerate}
\end{enumerate}

---

% 4
Determine convergence or divergence, including absolute/conditional convergence
if applicable. State clearly what test is being used.
\[\sum_{n=1}^\infty \frac{n!}{3^n}\]
===
(Ratio test) Because \(\lim_{n\to\infty}\left|\frac{n+1}{3}\right|=\infty>1\),
the series diverges.

---

% 5
Determine convergence or divergence, including absolute/conditional convergence
if applicable. State clearly what test is being used.
\[\sum_{n=1}^\infty \frac{2^n}{n^n}\]
===
(Ratio Test) Because \(\lim_{n\to\infty}\sqrt[n]{\frac{2^n}{n^n}}=0<1\), the
series converges.

---

% 6
Determine convergence or divergence, including absolute/conditional convergence
if applicable. State clearly what test is being used.
\[\sum_{n=1}^\infty \frac{1}{n\sqrt{n+1}}\]
===
(L.C.T.) Comparing against \(\sum_{n=1}^\infty \frac{1}{n^{\sfrac{3}{2}}}\)
(convergent p-series),
\[\lim_{n\to\infty}\frac{1}{n\sqrt{n+1}}\cdot\frac{n\sqrt{n}}{1}=1\]
Which is both finite and positive, and so converges similarly.

---

% 7
Determine convergence or divergence, including absolute/conditional convergence
if applicable; state clearly what test is being used; and determine the interval
of convergence for \(x\) if applicable.
\[\sum_{n=1}^\infty\frac{{(5x)}^n}{n^2}\]
===
The series converges for \(-\sfrac{1}{5}\leq x\leq\sfrac{1}{5}\).

---

% 8
Determine convergence or divergence, including absolute/conditional convergence
if applicable; state clearly what test is being used; and determine the interval
of convergence for \(x\) if applicable.
\[\sum_{n=1}^\infty\frac{{(-1)}^{n+1}x^{2n+1}}{(2n+1)!}\]
===
(Check for absolute convergence using Ratio Test)
Because \(\lim_{n\to\infty}\left|\frac{a_{n+1}}{a_n}\right|=0<1\), the series
converges absolutely for all values of \(x\).

---

% 9
Determine convergence or divergence, including absolute/conditional convergence
if applicable; state clearly what test is being used; and determine the interval
of convergence for \(x\) if applicable.
\[\sum_{n=1}^\infty\frac{(\ln n){(x-3)}^n}{n}\]
===
(Check for absolute convergence using Root test; check end points using A.S.T.
and L.C.T.) The series converges for \(2\leq x<4\).

---

% 10
Determine convergence or divergence, including absolute/conditional convergence
if applicable; state clearly what test is being used; and determine the interval
of convergence for \(x\) if applicable.
\[\sum_{n=1}^\infty n!{(n-1)}^n\]
===
(Ratio test) Because
\(\lim_{n\to\infty}\left|\frac{a_{n+1}}{a_n}\right|=\infty\), the series
diverges for all \(x\).

---

% 11
Given the alternating series
\[\sum_{n=1}^\infty\frac{{(-1)}^n}{2n^3+1}\]
\begin{enumerate}
  \item Use the first 4 terms to approximate the sum; then explain what the
    maximum error is in this approximation.
  \item Determine how many terms are required to estimate the sum with an error
    of no more than \(0.0001\).
\end{enumerate}
===
\begin{enumerate}
  \item \(S_4=-\frac{1}{3}+\frac{1}{17}-\frac{1}{55}+\frac{1}{129}\approx
    -0.2849\), and by alternating series remainder theorem,
    \(\text{error}=a_5=\left|\frac{1}{251}\right|\).
  \item
    \begin{gather*}
      a_{n+1}=\left|\frac{{(-1)}^n}{2n^3+1}\right|\leq 0.0001\\
      16.099\leq n\\
      n=17
    \end{gather*}

\end{enumerate}
