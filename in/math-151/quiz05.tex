::: Quiz 5

# 1
Does the series converge or diverge, and if it converges then find the sum (use
the geometric series test, the telescoping series test, or the nth term test for
divergence)
\[\sum_{n=1}^\infty\left(\frac{3}{4}\right)^n\]
===
Because \(r=|\sfrac{3}{4}|<1\) the series converges (geometric series), and
converges to the value 3.

---

# 2
Does the series converge or diverge, and if it converges then find the sum (use
the geometric series test, the telescoping series test, or the nth term test for
divergence)
\[\sum_{n=0}^\infty(1.2)^n\]
===
Because \(r=|1.2|>1\) the series diverges (geometric series).

---

# 3
Does the series converge or diverge, and if it converges then find the sum (use
the geometric series test, the telescoping series test, or the nth term test for
divergence)
\[\sum_{n=1}^\infty\frac{n}{n+3}\]
===
By the nth term test, \(\lim_{n\to\infty}\frac{n}{n+3}=1\neq 0\), therefore the
series diverges.

---

# 4
Does the series converge or diverge, and if it converges then find the sum (use
the geometric series test, the telescoping series test, or the nth term test for
divergence)
\[\sum_{n=1}^\infty\frac{1}{n(n+3)}\]
===
By means of partial fraction decomposition and evaluation of a telescopic
series, the series converges to the value \(\sfrac{11}{18}\).

---

# 5
Determine the convergence or divergence of the series using the \textit{integral
test}
\[\sum_{n=1}^\infty\frac{n}{n^2+1}\]
===
By the integral test, \(\lim_{b\to x}\int_1^b\frac{x}{x^2+1}dx=\infty\)
which is non-finite, therefore the series diverges.

---

# 6
Determine the convergence or divergence of the series using the \textit{limit comparison test}
\[\sum_{n=1}^\infty\frac{n}{n^2+1}\]
===
By the limit comparison test and selection of the harmonic series/divergent
p-series as the comparison series,
\[\displaystyle\lim_{n\to\infty}\frac{n}{n^2+1}\cdot\frac{n}{1}=1\]
Because the limit exists and is both positive and finite, the original series
behaves similarly to the comparison series, and thus diverges.

---

# 7

Find the explicit \textit{n}th term formula for the following sequence
\[\{a_n\}=\{-2,5,12,19,\dots\}\]
===
\[a_n=7n-9\]

---

# 8
Find the explicit \textit{n}th term formula for the following sequence
\[\{a_n\}=\{3,6,12,24,48,\dots\}\]
===
\[a_n=3(2)^{n-1}\]

---

# 9
Write the first four terms of the sequence
\[a_{n+1}=2a_n+1,\ a_1=1\]
===
\[\{1,3,7,15\}\]
