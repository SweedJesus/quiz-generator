::: Section 9.4

% Integral test

With \(a_n=f(n)\), \(\sum_{n=1}^\infty a_n\) and \(\int_1^\infty f(x)dx\)
either both diverge if \(f\) is \rule{1cm}{0.15mm}, \rule{1cm}{0.15mm}, and
\rule{1cm}{0.15mm} for \(x\geq 1\).
===
\underline{positive}, \underline{continuous}, and \underline{decreasing} for
\(x\geq 1\)

---

Use the integral test to determine if the following series converges or diverges
\[\sum_{n=1}^\infty \frac{1}{4n^2+1}\]
===
(Integral test) Because \(\int_1^\infty
\frac{1}{4x^2+1}dx=\frac{\pi}{4}-\frac{1}{2}\arctan(2)\approx 0.2318\)
(converges), the series also converges.

---

% Direct comparison test (DCT)

If \(\sum_{n=1}^\infty b_n\) converges and \(a_n\,\rule{1.25cm}{0.15mm}\,b_n\),
then \(\sum_{n=1}^\infty a_n\) also converges.
===
and \(a_n\leq b_n\)

---

If \(\sum_{n=1}^\infty b_n\) diverges and \(a_n\,\rule{1.25cm}{0.15mm}\,b_n\),
then \(\sum_{n=1}^\infty a_n\) also diverges.
===
and \(a_n\geq b_n\)

---

Use the direct comparison test to determine if the following series converges or
diverges
\[\sum_{n=1}^\infty \frac{1}{n!}\]
===
(D.C.T.) Comparing against \(b_n=\sum_{n=1}^\infty \frac{1}{2^n}\) (convergent
geometric series), because \(a_n<b_n\) the series similarly converges.

---

% Limit comparison test

Given a suitable \(a_n\) and \(b_n\) to compare against, the two series
\(\sum_{n=1}^\infty a_n\) and \(\sum_{n=1}^\infty b_n\) either both converge or
both diverge if
\(\lim_{n\to\infty}\left(\frac{\underline{?}}{\underline{?}}\right)=L\) where
\(L\) is both \rule{1cm}{0.15mm} and \rule{1cm}{0.15mm}.
===
\(\frac{\underline{a_n}}{\underline{b_n}}\); \underline{finite} and
\underline{positive}.

---

Use the limit comparison test to determine if the following series converges or
diverges
\[\sum_{n=1}^\infty \frac{\sqrt{n}}{n^2+1}\]
===
(L.C.T.) Comparing against \(\frac{1}{n^{3/2}}\) (convergent p-series), because
\(\lim_{n\to\infty}\frac{a_n}{b_n}=1\), which is finite and positive, the series
similarly converges.

---

% Alternating series

The series \(\sum_{n=1}^\infty {(-1)}^{n}\) and \(\sum_{n=1}^\infty
{(-1)}^{n+b}\) converge if the following two conditions are met:
\begin{enumerate}
  \item \(\lim_{n\to\infty} a_n=\underline{?}\)
  \item \(a_{n+1}\,\underline{?}\,a_n\) for all n
\end{enumerate}
===
\begin{enumerate}
  \item \(\lim_{n\to\infty} a_n=\underline{0}\)
  \item \(a_{n+1}\,\underline{\leq}\,a_n\) for all n
\end{enumerate}

---

% Absolute convergence

The series \(\sum_{n=1}^\infty a_n\) is \rule{1cm}{0.15mm} if
\(\sum_{n=1}^\infty |a_n|\) converges.
===
\underline{absolutely convergent}

---

The series \(\sum_{n=1}^\infty |a_n|\) is \rule{1cm}{0.15mm} if
\(\sum_{n=1}^\infty a_n\) converges, but \(\sum_{n=1}^\infty |a_n|\)
\rule{1cm}{0.15mm}.
===
\underline{conditionally convergent}; \underline{diverges}.

---

If you are given an alternating series:
\begin{enumerate}
  \item Check for \rule{1cm}{0.15mm} by applying a test on \(\sum_{n=1}^\infty
    |a_n|\)
  \item If the absolute value of the series \rule{1cm}{0.15mm}, then test for
    \rule{1cm}{0.15mm} using the \rule{1cm}{0.15mm}.
\end{enumerate}
===
\begin{enumerate}
  \item \underline{absolute convergence}
  \item \underline{diverges}; \underline{conditional convergence};
    \underline{alternating series test}.
\end{enumerate}

---

Determine if the series converges or diverges
\[\sum_{n=1}^\infty \frac{\sin n}{n^2}\]
===
(D.C.T.) Compared against \(\frac{1}{n^2}\) (convergent p-series), because
\(a_n\leq b_n\) the series similarly converges.

---

% 10.1---Taylor and Maclaurin Polynomials

If \(f\) has \(n\) derivatives at center \(a\), then the polynomial
\[P_n(x)=\rule{1cm}{0.15mm}+\cdots+\rule{1cm}{0.15mm}\]
is called the \textit{n}th degree Taylor polynomial for \(f\) at \(a\).
===
\[\underline{f(a)}+\cdots+\underline{\frac{f^{(n)}(a)}{n!}}\]
