::: Section 10.1

% 10.1---Taylor and Maclaurin Polynomials

If \(f\) has \(n\) derivatives at center \(a\), then the polynomial
\[P_n(x)=\rule{1cm}{0.15mm}+\cdots+\rule{1cm}{0.15mm}\]
is called the \textit{n}th degree Taylor polynomial for \(f\) at \(a\).
===
\[\underline{f(a)}+\cdots+\underline{\frac{f^{(n)}(a)}{n!}}\]

---

Find the \nth{10} degree Maclaurin Polynomial for
\[f(x)=\cos x\]
And then compare the value of \(\cos 0.5\) approximated by the polynomial with
the value given by a calculator (\(\approx 0.878\)).
===
\begin{gather*}
  P_{10}(x)=1-\frac{x^2}{2!}+\frac{x^4}{4!}-
  \frac{x^6}{6!}+\frac{x^8}{8!}-\frac{x^{10}}{10!}\\

  P_{10}(0.5)\approx 0.873
\end{gather*}

---

% Homework

Find the Taylor polynomials of degrees 0, 1, and 2, of the function \textit{f}
centered at point \textit{a}
\[f(x)=\ln(1+8x)\]
===
\begin{gather*}
  P_0(x)=0 \\
  P_1(x)=8x \\
  P_2(x)=8x-32x^2
\end{gather*}

---

Find the Taylor polynomials of degrees 0, 1, and 2, of the function \textit{f}
centered at point \textit{a}
\[f(x)=\cos x; a=\frac{\pi}{4}\]
===
\begin{gather*}
  P_0(x)=\frac{\sqrt{2}}{2}\\
  P_1(x)=\frac{\sqrt{2}}{2}-\frac{\sqrt{2}}{2}\left(x-\frac{\pi}{4}\right)\\
  P_2(x)=
  \frac{\sqrt{2}}{2}-
  \frac{\sqrt{2}}{2}\left(x-\frac{\pi}{4}\right)-
  \frac{\sqrt{2}}{4}{\left(x-\frac{\pi}{4}\right)}^2
\end{gather*}

---

Approximate \(e^{-0.06}\) using the Taylor polynomial
\[P_2(x)=1-x+\frac{x^2}{2}\]
===
\[e^{-0.06}\approx 0.9418\]
